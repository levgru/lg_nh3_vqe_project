\documentclass[10pt,twocolumn,letterpaper]{article}
\usepackage{geometry}
\usepackage{amsmath}
\usepackage{graphicx}
\usepackage{braket}
\usepackage{hyperref}

\title{Ammonia SDMP and Variational Quantum Eigensolvers}
\author{Lev Gruber}
\date{Spring 2024}

\begin{document}
\maketitle

\section{Background}

\subsection{Hybrid Algorithms: A Game of Hot Potato}
With the advent of the quantum computer came acknowledgement of both its strengths and weaknesses. They are spectacular for simulating and measuring our non-deterministic world, but not as great for processes such as optimization which is vital to a program's runtime and usage. 

In 2014, Alberto Peruzzo introduced the Variational Quantum Eigensolver (VQE), an algorithm that allows a classical and quantum computer to play to each of their respective strengths in the calculation of a system's ground state energy. In this way, the VQE plays a game of hot potato, tossing the heavy computational work between each computer type pending which is better suited for the task at hand. In this work, we discuss each piece of the VQE and how it may be applied to various physical problems; as a demonstration, we choose to demonstrate the VQE's use case through two simple molecules, dihydrogen ($H^2$) and ammonia ($NH^3$).  
\subsection{VQEs: A Breakdown}
\subsubsection{A Colloquial Breakdown}
VQEs work by mapping a hamiltonian, a large hermitian matrix, onto a quantum circuit then optimizing the parameters specifying this circuit using a classical computer. We call a circuit \textit{variational} when there are free parameters that can be adjusted to determine the circuit's properties. This is a \textit{quantum} algorithm because of its use of qubits on a quantum computer for the measurement of expectation values of observables. Finally, this is an \textit{eigensolver} meaning that its goal is to find the eigenvalues of our hamiltonian, those corresponding to the energies of our system. I use this algorithm to find the ground state energy of two different atomic systems.
\subsubsection{A Mathematical Breakdown}
Take an eigenvector of the hermitian hamiltonian (meaning $H = H^\dagger$) describing our atomic system and write the eigenvalue ($\lambda_i$) problem 
$$H\ket{\psi_i}=\lambda_i\ket{\psi_i} \text{with } H = \displaystyle \sum_{i=1}^N\lambda\ket{\psi_i}\bra{\psi_i}.$$
We want to measure expectation values on our quantum computer \cite{vqe}, so consider the relation between minimum eigenvalue and expectation value. Take the below equation and plug in the series representation of $H$ to get
$$\langle H \rangle_\psi = \langle \psi | H | \psi \rangle = \displaystyle \sum_{i=1}^N \lambda_i | \langle \psi_i | \psi_i \rangle|^2 $$ meaning that $\lambda_{min} \leq \langle \psi | H \rangle_\psi$. This is known as the \textit{variational method/principle}, and gives that the expectation value of any wavefunction is always greater than or equal to $H$'s minimum eigenvalue.

With our VQE we want to find the minimum energy of a system, which here translates to finding the minimum eigenvalue of our system's hamiltonian, which we've shown can be done so through the minimization of the wavefunction's expectation value. In a VQE's case, the \textit{ansatz}'s--initial guess of $\ket{\psi_{min}}$'s expectation value can be estimated by taking a quantum circuit with action $U(\theta)$ where $\theta$ is our variational parameter and applying it to an initial $\ket{\psi_i}$, optimizing over $\psi(\theta)$ until we achieve a $\lambda_{min}$ corresponding to $\langle \psi | H \rangle_\psi$. Recall that $\lambda_{min}$ is our ground state energy--with this background we can explore the applications of a VQE.
\section{Application}
This paper demonstrates two separate but connected examples of a variational quantum eigensolver in use. The first entailed an incredibly simple molecule, dihydrogen ($H_2$), and the determination of its ground state energy at various interatomic distances. The second explored quantum tunneling possible in ammonia ($NH_3$) through finding ground state energy at various distances of nitrogen from the planar hydrogens. See figure 1 for an image of the geometries involved. 
\begin{figure}[h]
    \begin{minipage}{0.3\textwidth}
        \includegraphics[width=\linewidth]{h2_im.png}
        \label{1}
    \end{minipage}
    \hfill
    \begin{minipage}{0.5\textwidth}
        \includegraphics[width=\linewidth]{nh3_im.png}
    \end{minipage}
    \caption{Dihydrogen ($H_2$, left) molecule and ammonia in two symmetric states ($NH_3$, right).}
\end{figure}
\subsection{$H_2$}
The dihydrogen molecule's ground state energy changes as the interatomic distance between the two hydrogen's increase\cite{vqe_vid}--as such, we can use the variational quantum eigensolver at various interatomic distances to characterize how the two are related. The ansatz for the VQE is the Unitary Coupled-Cluster Singles and Doubles (UCCSD) ansatz, which put simply models the single and double excitation of electrons while maintaining the unitary nature of the operators involved \cite{qiskit_nature}. The initial state is determined by the commonly used Hartree-Fock Method, meaning we approximate the many electron wave function as a single Slader determinant, called the Hartree-Fock determinant. The UCCSD method then takes the Hartree-Fock determinant and applies a unitary transformation to it, creating correlated wave functions with which we can run our VQE--this is a very simplified explanation and more can be learned in \cite{qiskit_nature}.

Plotted below Figure $2$ is the hartree fock method (blue), exact solution which is attained using the NumPy eigensolver (orange), and VQE solutions (green). It is shown that the VQE had great success in simulating the ground state energies of hydrogen at these interatomic distances. 
\subsection{$NH_3$}
The ammonia molecule has a symmetric double minima potential (SDMP), where potential energy minima occur in two different molecular states \cite{nh3}. As in Figure 1, the ammonia form a trigonal pyramidal structure where the nitrogen atom can be found in a minimum energy state above or below the planar hydrogens. As the measured observable (ground state energy) for each state must be the same, $$\langle \text{below} | H | \text{below} \rangle = \langle \text{above} | H | \text{above} \rangle = E_0$$
These two states are separated by a high but not infinite energy barrier. In a classical system, this energy barrier would be insurmountable, but due to quantum mechanical effects there is a \textit{Transmission coefficient} $T$ that is non-zero and provides that there is a non-zero probability that the nitrogen atom tunnels through the planar hydrogens from one ground state energy configuration to the other \cite{townsend}. A VQE can be used to find this energy barrier and the minima, as is done in figure 3. To make this problem computationally tractable, we define the complete active space (CAS) of the molecule to have two electrons and two spatial orbitals--this means that we are assuming that the majority of the ground state energy contribution stems from these two electrons and orbitals, such that we only run the VQE on this CAS while determining energy contribution from the rest of the system in a much simpler way \cite{vqe}. This brings the problem's complexity nearer to that of dihydrogen.

The ground state energy of the ammonia molecule was found computationally with two methods: using a simulation provided by the Qiskit package \cite{Qiskit}, and using one of IBM's quantum computers located in Kyoto, Japan \cite{ibm}. The first provides both noiseless (local simulator) and noisy (cloud simulator) results--the noiseless simulation is useful for validation of the VQE method while noisy helps characterize quantum results and provide checks against randomness as we expect from anything calculated on a modern quantum computer. The second is useful due to it showing the utility, prospects, and shortcomings of an actual quantum computer. 

On the following page, see figures $3$ and $4$. Figure $4$ shows the ammonia SDMP when calculated exactly, with a noisy simulator, and with a noiseless quantum computer simulator and variational quantum eigensolver. Despite an apparent shift in energy, the trend of a symmetric double minima and a single maxima through which the nitrogen atom could tunnel is apparent and allows us to determine potential energy barrier's scale. The two minima correspond to the states shown in Figure $1$, while the middle maxima corresponds to the nitrogen being in middle of the planar hydrogens. We also see that most points of the noisy simulator align with the locally simulated VQE result as well as (energy shifted) the exact result. This was expected, as introducing noise in turn introduces randomness on a quantum computer. Figure $3$ plots the same data as $4$ while introducing results from IBM's quantum computer in Kyoto, Japan, which was accessed through IBM's quantum experience program. These three points took 30 minutes of combined quantum computation time and 27 hours of running code (due to long wait times for quantum computer use). Despite an extreme amount of noise which is to be expected from noisy intermediate-scale quantum computers (NISQ), which have a lot of noise and short coherence times. Despite this, we do still observe the expected pattern of a maxima and two minima, albeit at a much higher energy level than both the exact and simulated results. This study would be benefitted from greater access to IBM's quantum computer's and potentially more optimization of the quantum circuit provided to IBM which would achieve better run times and more attainable points.

\subsection{Further Applications}
The above sections showcase the results of this project; all code, some figures, and further resources can be found in the bibliography below. There are many applications of a variational quantum eigensolver outside of the examples presented here. Some to explore further include modeling wormholes with the SYK hamiltonian, complicated chemistry modeling for drug discovery, characterizing superconductors, and finding enzyme structure in molecular biology. The key takeaway, and the VQEs best use case, lies in simplifying computationally expensive problems to something easily measurable by a quantum computer, therefore making an impossible problem fully tractable. This algorithm will only become more powerful as the technology grows and quantum computers become able to handle more qubits and gate operations with less noise; this necessitates future research in quantum error correction, allowing us to make more accurate observations about our universe. 

\begin{figure*}[t]
    \begin{minipage}{0.5\textwidth}
        \includegraphics[width=\linewidth]{hh-plot.png}
        \caption{Hydrogen ground state energy as a function of interatomic distance}
        \label{2}
    \end{minipage}
    \begin{minipage}{0.5\textwidth}
        \includegraphics[width=1.1\linewidth]{nh3_real.png}
        \caption{Ammonia SDMP, quantum computed}
        \label{3}
    \end{minipage}
\end{figure*}
\begin{figure*}[!]
    \includegraphics[width=\linewidth]{nh3_sim.png}
        \caption{Ammonia SDMP, simulated}
        \label{4}
\end{figure*}

\newpage
\bibliographystyle{plain}
\bibliography{bib64}

\nocite{*}

\end{document}